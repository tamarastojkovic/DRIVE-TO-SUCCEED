\section{Slučajevi upotrebe}
\label{sec:podnaslov2}
U ovoj sekciji su predstavljeni slučajevi upotrebe (3 glavna, i slučajevi u okviru njih).
Za svaki slučaj upotrebe predstavljen je i odgovarajući dijagram aktivnosti, 
u okviru sekcije za taj pojedinačni slučaj upotrebe.
Za svaki glavni slučaj dodat je dijagram tog slučaja upotrebe, kao i odgovarajući BPMN dijagrami.
\subsection {Podnošenje zahteva za prijavu}
Proces registracije korisnika u auto školi započinje popunjavanjem online formulara, gde kandidat unosi svoje lične podatke. Administrativni radnik treba da proveri da kadnidat ispunjava potrebne uslove za upis i formira neophodnu dokumentaciju za kandidata. Nakon toga informacije prosleđuje administratoru sistema, koji unosi novog korisnika u bazu i prosleđuje radniku ID i lozinku za trenutnog korisnika. Kada je kandidat prijavljen u auto školu, dobija imejl sa potvrdom o registraciji, svoj ID i lozinku , pa se može ulogovati na svoj nalog. 

\begin{figure}[H]
    \begin{center}
        \includegraphics[width=120mm, height=60mm]{Diagrams/podnosenje_zahteva_za_prijavu.png}
    \end{center}
    \caption {Podnošenje zahteva za prijavu}
    \label{usecase_podnosenje_zahteva_za_prijavu}

\end{figure}



\subsubsection{Podnošenje prijave}
\label{subsubsec:prijava}
\begin{itemize}
  \item \textbf{Kratak opis}: Da bi kandidat započeo obuku u auto školi, prvo mora da podnese prijavu za upis. Popunjava online formular, gde unosi svoje lične podatke, koji se nakon potvrde šalju administrativnom radniku. Radnik formira dokumentaciju za kandidata i prosleđuje ih administratoru sistema, koji unosi informacije o kandidatu u sistem.
  \item \textbf{Učesnici}: 
    \begin{itemize} 
      \item Kandidat
      \item Administrativni radnik
      \item Administrator sistema
    \end{itemize} 
  \item \textbf{Preduslovi}:
    \begin{itemize}
    \item Kandidat mora posedovati važeću ličnu kartu.
    \item Kandidat mora uplatiti prvu ratu.
    \end{itemize}
  \item \textbf{Postuslovi}:
      \begin{itemize}
      \item Kandidat ispunjava uslove za prijavu.
      \end{itemize}
  \item \textbf{Osnovni tok}:
      \begin{enumerate}
        \item Kandidat otvara online formu za prijavu u auto školu.
        \item Sistem prikazuje formu kandidatu.
        \item Kandidat popunjava formu, unoseći sve potrebne informacije.
        \item Kanidat šalje unete podatke klikom na dugme “Pošalji”.
        \item Sistem validira unos podataka.
        \item Sistem čuva podatke.
        \item Sistem prosleđuje podatke administrativnom radniku.
        \item Аdministrativni radnik proverava da li kanidat ispunjava uslove za upis.
        \item Administrativni radnik formira dokumentaciju za kandidata.
        \item Administrativni radnik prosleđuje podatke o kandidatu administratoru sistema.
      \end{enumerate}

  \item \textbf{Alternativni tokovi}:
      \begin{itemize}
        \item A1. \textbf{Kandidat nije uneo ispravne podatke.}
        Ukoliko je u koraku 5 sistem uočio nevalidnost u formatu unetih podataka od strane kandidata, neophodno je da ih kandidat unese ponovo u ispravnom obliku.
        Proces se nastavlja u koraku 1 osnovnog toka.
        \item A2. \textbf{Kandidat ne ispunjava uslove za upis.}
        Ukoliko je u koraku 8 administrativni radnik uočio da kandidat ne ispunjava uslove za prijavu, kontaktira ga kako bi ga obavestio o tome.
        Kandidat se ne može upisati u auto školu i proces se završava.
      \end{itemize}

  \item \textbf{Dodatne informacije}:\newline
  Potrebni podaci za prijavu su ime, prezime, JMBG, broj telefona, imejl adresa i potvrda o uplati prve rate.
\end{itemize}

\begin{figure}[H]
  \begin{center}
      \includegraphics[width=140mm, height=70mm]{Diagrams/dijagram_aktivnosti_podnosenje_prijave.png}
  \end{center}
  \caption {Dijagram aktivnosti - Podnošenje prijave}
  \label{activity_podnosenje_prijave}

\end{figure}





\subsubsection{Registracija kandidata}
\label{subsubsec:registracija}
\begin{itemize}
  \item \textbf{Kratak opis}: Da bi kandidat mogao da se uloguje na svoj nalog i vidi svoje podatke, 
  nepohodno je da dobije potvrdu da je upisan u auto školu, kao i nepohodan ID i lozinku za logovanje.
  \item \textbf{Učesnici}:
  \begin{itemize}
    \item Kandidat
    \item Administrator sistema
    \item Administrativni radnik.
  \end{itemize}
  \item \textbf{Preduslovi}:
    \begin{itemize}
    \item  Kandidat je popunio online prijavu.
    \item  Kandidat ispunjava uslove za upis.
    \end{itemize}
  \item \textbf{Postuslovi}:
      \begin{itemize}
      \item Kandidat je evidentiran u sistemu.
      \item Kandidat može da se prijavi na sistem.
      \end{itemize}
  \item \textbf{Osnovni tok}:
      \begin{enumerate}
        \item Administrator sistema prima informacije o novom kadnidatu.
        \item Administrator sistema unosi novog korisnika u bazu podataka.
        \item Sistem čuva unete podatke.
        \item Sistem šalje mejl novom kandidatu sa potvrdom o registraciji i podacima.
        \item Kandidat dobija mejl sa potvrdom o registraciji i podacima.
        \item Sistem šalje mejl administrativnom radniku da je uspešno dodao novog kanidata.
        \item Administrativni radnik prima mejl da je kandidat dodat u bazu.
        \item Administrativni radnik ažurira spisak prijavljenih kandidata dodavanjem novog kadnidata.    
      \end{enumerate} 

      \newpage
  \item \textbf{Alternativni tokovi}:
      \begin{itemize}
        \item A1. \textbf{Kandidat nije dobio mejl sa ID-jem i šifrom za pristupanje svom nalogu.}
        Ukoliko u koraku 5 kandidat nije dobio mejl, administrator sistema zahteva od sistema da ponovo pošalje mejl. Proces se nastavlja u koraku 4. osnovnog toka.
        \item A2. \textbf{Administrativni radnik nije dobio potvrdu o dodavanju novog kandidata.}
        Ukoliko u koraku 7 administrativni radnik nije dobio potvrdu o dodavanju novog kandidata, mejl mu se ponovo šalje tj. proces se nastavlja od 6. koraka osnovnog toka.
      \end{itemize}


  \item \textbf{Dodatne informacije}:\newline
  Podaci koji se prosleđuju kandidatu putem mejla su njegov ID i lozinka u bazi, kako bi mogao da se uloguje.
\end{itemize}

\begin{figure}[H]
  \begin{center}
      \includegraphics[width=140mm, height=70mm]{Diagrams/dijagram_aktivnosti_registracija_kandidata.png}
  \end{center}
  \caption {Dijagram aktivnosti - Registracija kandidata}
  \label{activity_registracija}

\end{figure}


\subsubsection{Raspoređivanje kandidata u grupu}
\label{subsubsec:grupe}
\begin{itemize}
  \item \textbf{Kratak opis}: Da bi kandidat bio raspoređen u grupu neophodno je da odabere neku od ponuđenih grupa, nakon logovanja na svoj nalog. 
  \item \textbf{Učesnici}: 
    \begin{itemize}
    \item  Kandidat
    \end{itemize}
  \item \textbf{Preduslovi}:
    \begin{itemize}
    \item Kandidat je dobio svoj ID i lozinku na mejl.
    \end{itemize}
  \item \textbf{Postuslovi}:
      \begin{itemize}
      \item Kandidat je raspoređen u neku grupu.
      \end{itemize}
  \item \textbf{Osnovni tok}:
      \begin{enumerate}
        \item Kandidat se prijavljuje sa svojim ID-jem i šifrom na nalog.
        \item Kandidat klikom na dugme “Grupe” bira neku od ponuđenih grupa.
        \item Sistem šalje mejl kandidatu da je uspešno raspoređen u grupu i raspored održavanja časova.
        \item Kandidat dobija mejl sa podacima o grupi i rasporedu nastave.
      \end{enumerate}

  \item \textbf{Alternativni tokovi}:
      \begin{itemize}
        \item A1. \textbf{Neuspešno prijavljivanje.}
        Kandidat je pogrešio ID ili šifru u koraku 1, pa je nepohodno da proveri ispravnost podataka i da ih ponovo unese. Proces se nastvalja od koraka 1. osnovnog toka.
        \item A2. \textbf{Kandidat nije dobio mejl o raspoređivanju po grupama.}
        Kandidat nije raspoređen u željenu grupu, jer nije dobio mejl sa potvrdom u koraku 4, možda jer je vreme za prijavu isteklo (neko drugi je odabrao preostalo mesto). Proces se nastavlja od koraka 1. osnovnog toka.
      \end{itemize}
\end{itemize}

\begin{figure}[H]
  \begin{center}
      \includegraphics[width=140mm, height=70mm]{Diagrams/dijagram_aktivnosti_grupe.png}
  \end{center}
  \caption {Dijagram aktivnosti - Reaspoređivanje kandidata u grupu}
  \label{activity_grupe}

\end{figure}

\begin{figure}[H]
    \begin{center}
        \includegraphics[width=120mm, height=60mm]{Diagrams/bpmn_podnosenje_zahteva.png}
    \end{center}
    \caption {BPMN dijagram procesa podnošenja zahteva za prijavu}
    \label{bpmn_podnosenje_zahteva}

\end{figure}


\subsection {Teorijska nastava}
Proces prijave za teorijsku nastavu kandidata, kao i vodjenje evidencije casova se odvija u okviru teorijske nastave. Takdoje, tu imamo i evidenciju polaganja kandidata koji su uspesno prijavili svoj ispit i odslusali casove predavanja.

\subsubsection{Teorijska nastava}
\label{subsubsec:prijava za nastavu}
\begin{itemize}
  \item \textbf{Kratak opis}: Kandidat koji se upisao u auto školu počinje sa pohađanjem teorijske nastave u izabranoj grupi. 
  Predavač evidentira prisustvo kandidata na časovima teorije koje on drži.
  \item \textbf{Učesnici}:
    \begin{itemize}
    \item  Kandidat - korisnik sistema koji pohađa nastavu.
    \item  Predavač - korisnik koji drži časove i evidentira prisustvo.
    \end{itemize}
  \item \textbf{Preduslovi}:
    \begin{itemize}
    \item  Kandidat mora biti upisan u auto školu.
    \item  Kandidat mora biti raspoređen u grupu kod predavača.
    \item  Kandidat je izmirio  prethodne troškove upisa.
    \item  Predavač je zadužen za grupu koju kandidati pohađaju.
    \item  Predavač je ulogovan na sistem.
    \item  Sistem je u funkciji.
    \item  U sistemu ne postoji evidentiran čas za grupu u tekućem danu.
    \item  Predavač  ima pristup internetu.
    \end{itemize}
  \item \textbf{Postuslovi}:
      \begin{itemize}
      \item Kandidat je evidentiran da je pohađao nastavu.
      \item Predavač je evidentirao održano predavanje.
      \end{itemize}
  \item \textbf{Osnovni tok}:
      \begin{enumerate}
        \item Predavač otvara stranicu za evidenciju prisustva korisnika.
        \item Predavač popunjava formular za započinjanje časa sa grupom.
        \item Predavač potvrdjuje da zapocinje cas sa grupom.
        \item Sistem prikazuje listu kandidata koji pohađaju nastavu u toj grupi.
        \item Predavač evidentira prisustvo za svakog kandidata.
        \item Predavač zaključuje evidenciju.
        \item Predavač započinje predavanje.
        \item Predavač nakon održanog časa zaključuje čas.
        \item Sistem šalje mail svim učesnicima o uspešno završenom času i njihovom napretku. %ovo treba obrisati
      \end{enumerate}

  \item \textbf{Alternativni tokovi}:
      \begin{itemize}
        \item A1. \textbf{Neuspela validacija.}Ukoliku u koraku 2 sistem pronalazi neispravno polje formulara sistem obeležava polje koje treba ispraviti 
        crvenom bojom, a ispod polja piše  uzrok neispravnosti. Nakon ponovnog ispravnog unosa podataka proces se nastavlja u koraku 3 osnovnog toka.
      \end{itemize}
      
 \item \textbf{Specijalni zahtevi}:
      \begin{itemize}
        \item Polja formulara pri započinjanju časa su: grupa, termin, čas.
      \end{itemize}
\end{itemize}

\begin{figure}[H]
  \begin{center}
      \includegraphics[width=140mm, height=70mm]{Diagrams/teorijska nastava dijagram.png}
  \end{center}
  \caption {Dijagram aktivnosti - teorijska nastava}
  \label{activity_diagram}

\end{figure}


\subsubsection{Prijava za teorijski ispit}
\label{subsubsec:prijava za ispit}
\begin{itemize}
  \item \textbf{Kratak opis}: Kandidat nakon završenog pohađanja časova teorije podnosi prijavu za polaganje teorijskog ispita
  \item \textbf{Učesnici}:
    \begin{itemize}
    \item Kandidat - korisnik sistema koji se prijavljuje za ispit.
    \end{itemize}
  \item \textbf{Preduslovi}:
    \begin{itemize}
    \item  Kandidat mora biti upisan u auto školu.
    \item  Kandidat je odslušao sve casove teorije.
    \item  Kandidat je izmirio prethodne troškove prijave.
    \item  Kandidat je ulogovan na sistem.
    \item  Sistem je dosutpan.
    \item  Kandidat ima pristup internetu.
    \end{itemize}
  \item \textbf{Postuslovi}:
      \begin{itemize}
      \item Kandidat je podneo prijavu za polaganje teorijskog ispita.
      \end{itemize}
  \item \textbf{Osnovni tok}:
      \begin{enumerate}
        \item Kandidat otvara stranicu za prijavu polaganja teorijskog ispita.
        \item Sistem prikazuje formular za prijavljivanje teorijskog ispita.
        \item Kandidat popunjava formular.
        \item Kandidat potvrđuje prijavu klikom na dugme.
        \item Sistem evidentira prijavu.
        \item Sistem šalje mail kandidatu o uspesnoj prijavi.  
      \end{enumerate}

  \item \textbf{Alternativni tokovi}:
      \begin{itemize}
        \item A1. \textbf{Neuspela validacija.}
        Ukoliku u koraku 3 sistem pronalazi neispravno polje formulara sistem obeležava polje koje treba ispraviti crvenom bojom, 
        a ispod polja piše uzrok neispravnosti. Nakon ponovnog ispravnog unosa podataka proces se nastavlja u koraku 5.
      \end{itemize}
      
  \item \textbf{Specijalni zahtevi}:
      \begin{itemize}
        \item Polja formulara za prijavu: Ime, Prezime, JMBG, Datum poslednjeg časa, Predavač, Skenirana lična karta 
      \end{itemize}
\end{itemize}

\begin{figure}[H]
  \begin{center}
      \includegraphics[width=140mm, height=70mm]{Diagrams/prijava za polaganje teorijskog ispita.png}
  \end{center}
  \caption {Dijagram aktivnosti - prijava za teorijski ispit}
  \label{activity_prijava_za_teoriju}

\end{figure}


\subsubsection{Izlazak na teorijski ispit}
\label{subsubsec:teorijski ispit}
\begin{itemize}
  \item \textbf{Kratak opis}: Kandidat koji je uspešno prijavio teorijski ispit izlazi na polaganje.
  \item \textbf{Učesnici}:
    \begin{itemize}
    \item Kandidat - korisnik sistema koji polaže ispit.
    \end{itemize}
  \item \textbf{Preduslovi}:
    \begin{itemize}
    \item  Kandidat mora biti upisan u auto školu.
    \item  Kandidat je odslušao sve časove teorije.
    \item  Kandidat je izmirio prethodne troškove prijave.
    \item  Kandidat je ulogovan na sistem.
    \item  Kandidat je uspešno prijavio teorijski ispit.
    \item  Sistem je u funkciji.
    \item  Kandidat ima pristup internetu.
    \end{itemize}
  \item \textbf{Postuslovi}:
      \begin{itemize}
      \item  Kandidat je je završio pohađanje teorijskog ispita.
      \end{itemize}
  \item \textbf{Osnovni tok}:
      \begin{enumerate}
        \item Kandidat je otvorio stranicu za polaganje teorijskog ispita.
        \item Sistem šalje na mail pristupnu lozinku za ispit.
        \item Kandidat unosi pristupnu lozinku.
        \item Sistem otvara stranicu sa teorijskim ispitom za kandidata.
        \item Kandidat potvrđuje da hoće da završi izradu ispita (ili je isteklo vreme za izvršavanje ispita).
        \item Sistem otvara stranicu sa rezultatima polaganja.
        \item Sistem šalje kandidatu mail sa ishodom polaganja za prijavu i rezultatima.
      \end{enumerate}

  \item \textbf{Alternativni tokovi}:
      \begin{itemize}
        \item A1. \textbf{Neuspela provera koda.}
        Neuspela provera koda: Ukoliko u koraku 3 kandidat unese loš pristupni kod polje za kod će postati crveno,
         i biće mu omogućeno da ponovo unese kod, ili da ponovno pošalje kod na mail. Kada ispravno unese kod proces se nastavlja u koraku 4.
      \end{itemize}
      
  \item \textbf{Specijalni zahtevi}:
      \begin{itemize}
        \item Polja formulara za prijavu: pristupni kod. 
      \end{itemize}
\end{itemize}

\begin{figure}[H]
  \begin{center}
      \includegraphics[width=140mm, height=70mm]{Diagrams/polaganje teorijskog ispita.png}
  \end{center}
  \caption {Dijagram aktivnosti - polaganje teorijskog ispita}
  \label{activity_polaganje_teorije}

\end{figure}


\subsection {Praktična nastava}
Proces praktične obuke započinje dostavljanjem auto školi potvrde o obavljenom lekarskom pregledu, koji kandidat mora da obavi. Neophodno je da kandidat odabere jednog od instruktora. Nakon 40 časova vožnje, i dobijene potvrde o položenoj prvoj pomoći, kandidat stiče uslov za prijavu za izlazak na vozački ispit. Kandidat se prijavljuje za ispit u jednom od ponuđenih termina i neophodno je da ima evidentirane potrebne uplate. Ukoliko je položio vozački ispit, pre izdavanja potvrde o položenom ispitu, neophodno je da kandidat popuni anketu o auto školi. Ovo će pomoći novim kandidatima, pre sve pri izboru odgovarajućeg instruktora.

\begin{figure}[H]
    \begin{center}
        \includegraphics[width=107mm, height=60mm]{Diagrams/prakticna_obuka.png}
    \end{center}
    \caption {Praktična obuka}
    \label{usecase_praktična obuka}

\end{figure}

\subsubsection{Prijava za praktičnu obuku}

\vspace{3mm}

\begin{itemize}

\item \textbf{Kratak opis:} Da bi kandidat započeo praktičnu obuku potrebno je da se prijavi. Pre prijave kandidat dostavlja auto školi potvrde o obavljenom lekarskom pregledu, uplatama, nakon čega bira instruktora.

\vspace{2mm}

\item \textbf{Učesnici} \newline
   - Kandidat \newline 
   - Instruktor \newline 
   - Administrativni radnik.

\item \textbf{Preduslovi:} \newline
   - Kadidat je položio teorijski ispit. \newline 
   - Kandidat je obavio lekarski pregled. 

\item \textbf{Postuslovi:} \newline
    - Kandidat se prijavio za praktičnu obuku 

\item \textbf{Osnovni tok:}  
   \begin{enumerate}
   \item Kandidat je dosatvio administrativnom radniku svu neophodnu dokumentaciju.
   \item Administrativni radnik je validirao ispravnost i potpunost dokumentacije.
   \item Kandidat bira željenog instruktora.
   \item Sistem potvrđuje kandidatu prijavu
   \end{enumerate}

\item \textbf{Alternativni tok:}  
   \begin{itemize}
   \item A1. \textbf{Neispravnost ili nepotpunost dokumentacije:}
  Dokumentacija koju je korinik priložio je nepotpuna ili sadrži neke neispravne podatke. Slučaj upotrebe se privremeno zaustavlja dok kandidat ne kompletira potrebnu dokumentaciju ili ispravi prosleđene podatke i proces se nastavlja od koraka 1 u osnovnom toku.
   \end{itemize}

\end{itemize}  


\subsubsection{Pohađanje praktične obuke}

\vspace{3mm}

\begin{itemize}

\item \textbf{Kratak opis:} Kandidat sa instruktorom dogovara časove vožnje. Neophodno je da kandidat prisustvuje ukupno 40 časova i prisustvo se belezi u sistem.

\vspace{2mm}

\item \textbf{Učesnici} \newline
   - Kandidat - korisnik sistema koji pohađa praktičnu obuku.\newline   
   - Instruktor - korisnik sistema koji drži praktičnu obuku. 
   
\item \textbf{Preduslovi:} \newline
   - Sistem je u funkciji. \newline
   - Kandidat je prijavljen za praktičnu obuku 

\item \textbf{Postuslovi:} \newline
    - Kandidat je završio praktičnu obuku.

\item \textbf{Osnovni tok:}  
   \begin{enumerate}
   \item Kandidat otvara formular za zakazivanje časa.
   \item Sistem prijazuje formular.
   \item Kandidat zakazuje čas.
   \item Instruktor potvrđuje čas.
   \item Kandidat prisustvuje času u odgovarajućem terminu.
   \item Instruktor otvara formular za evidenciju časa.
   \item Sistem prikazuje formular.
   \item Instruktor unosi podatke o održanom času.
   \item Kandidat potvrđuje podatke o održanom času.
   \item Sistem čuva unete podatke. \newline
   \end{enumerate}

\item \textbf{Alternativni tok:}  
   \begin{itemize}
   \item A1. \textbf{Kandidat otkazuje čas:}
  Kandidat obaveštava instruktora da ne može da prisustvuje času i zakazuje novi termin časa, proces se nastavlja u koraku 1 osnovnog toka.
   \end{itemize}

\item \textbf{Dodatne informacije:}    
\begin{itemize}
\item Osnovni tok se ponavlja 40 puta. Nakon toga, sistem šalje mejl kandidatu sa potvrdom o žavršenoj praktičnoj obuci.
\end{itemize}

\end{itemize}

\begin{figure}[H]
  \begin{center}
      \includegraphics[width=140mm, height=70mm]{Diagrams/dijagram_aktivnosti_pohadjanje_prakticne_obuke.png}
  \end{center}
  \caption {Dijagram aktivnosti - Pohađanje praktične obuke}
  \label{activity_pohadjanje_prakticne_obuke}

\end{figure}


\subsubsection{Prijava za praktični ispit}

\vspace{3mm}

\begin{itemize}

\item \textbf{Kratak opis:} Da bi kandidat izašao na vozački ispit prvo podnesi prijavu za polaganje. Popunjava online formular gde bira željeni termin, koji se  šalje administrativnom radniku. Nakon potvrde termina administrativni radnik šalje mejl sa terminom ispita i osnovnim info.

\vspace{2mm}

\item \textbf{Učesnici} \newline
   - Kandidat \newline   
   - Administrativni radnik 
   
\item \textbf{Preduslovi:} \newline
   - Kandidat je položio prvu pomoć. \newline
   - Kandidat je završio praktičnu obuku. \newline
   - Kandidat je izvršio sve uplate

\item \textbf{Postuslovi:} \newline
    - Kandidat je prijavljen za izlazak na vozački ispit.

\item \textbf{Osnovni tok:}  
   \begin{enumerate}
   \item Kandidat otvara online formular za prijavu za izlazak na ispit.
   \item Kandidat bira termin za ispit.
   \item Kandidat se prijavljuje klikom na dugme "Pošalji".
   \item Administrativni radnik proverava da li kandidat ispunjava uslove za izlazak.
   \item Administrativni radnik šalje mejl kandidatu gde potvrđuje prijavu i dostavlja dodatne informacije.
   \item Kandidat otvara mejl i proverava da li je dobio potvrdu za izlazak na ispit. 
   \item Sistem beleži podatke o prijavi. 
   \end{enumerate}

\item \textbf{Alternativni tok:}  
   \begin{itemize}
   \item A1. \textbf{Nevalidni podaci:}
  Kandidat ne ispunjava uslove za prijavu na ispit. Proces se prekida dok kandidat ne ispuni sve uslove za izlazak na ispit.
  \item A2. \textbf{Kandidat nije dobio potvrdu za izlazk na ispit:}
  Administrativni radnik nije poslao potvrdu kandidatu. Kandidat obaveštava administrativnog radnika da nije dobio mejl. Proces se nastavlja u koraku 5 osnovnog toka.
   \end{itemize}

\end{itemize}  

\begin{figure}[H]
  \begin{center}
      \includegraphics[width=140mm, height=70mm]{Diagrams/dijagram_aktivnosti_prijava_za_praktican_ispit.png}
  \end{center}
  \caption {Dijagram aktivnosti - Prijava za praktični ispit}
  \label{activity_prijava_za_prakticni_ispit}

\end{figure}


\subsubsection{Polaganje praktičnog ispita}

\vspace{3mm}

\begin{itemize}

\item \textbf{Kratak opis:} Kandidat polaže vozački ispit i nakon uspešnog polaganja popunjava anketu o auto školi i dobija potvrdu o položenom vozačkom ispitu.

\vspace{2mm}

\item \textbf{Učesnici} \newline
   - Kandidat \newline   
   - Instruktor \newline
   - Administrativni radnik 
   
\item \textbf{Preduslovi:} \newline
   - Sistem je u funkciji. \newline
   - Kandidat se prijavio za praktični ispit.

\item \textbf{Postuslovi:} \newline
    - Kandidat je završio sa obukom.

\item \textbf{Osnovni tok:}  
   \begin{enumerate}
   \item Kandidat izlazi na završni ispit.
   \item Sistem prikazuje instruktoru putanju vožnje za tog kandidata.
   \item Instruktor saopštava putanju kandidatu.
   \item Kandidat vozi po datoj putanji.
   \item Kandidat završava ispit.
   \item Instruktor saopštava rezultate ispita.
   \begin{enumerate}
       \item Ukoliko je kandidat položio ispit, izvršava se podtok P1.
       \item Ukoliko kandidat nije položio ispit, izvršava se podtok P2.
   \end{enumerate}
   \item Sistem ažurira podatke o kandidatu.     

   \end{enumerate}

\item \textbf{Podtokovi:}   
 \begin{itemize}
   \item P1. \textbf{Kandidat je položio ispit:}
   \begin{enumerate}
       \item Instruktor unosi podatke o ispitu u sistem.
       \item Instruktor potvrđuje položen ispit u sistemu.
       \item Sistem čuva podatke o ispitu.
       \item Sistem šalje kandidatu anketu.
       \item Kandidat popunjava anketu.
       \item Sistem čuva podatke iz ankete.
       \item Administrativni radnik šalje kandidatu potvrdu o položenom vozačkom ispitu.
       \item Kandidat ulazi na mejl i proverava potvrdu o položenom vozačkom ispitu.
   \end{enumerate}
   \item P2. \textbf{Kandidat nije položio ispit:}
   \begin{enumerate}
       \item Instruktor unosi podatke o ispitu u sistem.
       \item Sistem čuva podatke o ispitu.
       \item Proces se prekida dok se kandidat opet ne prijavi za polaganje prakticnog ispita.
   \end{enumerate}
  
   \end{itemize}

\item \textbf{Alternativni tok:}  
   \begin{itemize} 
     \item A1. \textbf{Kandidat nije dobio potvrdu o položenom ispitu:}
     Ukoliko u koraku 8, podtoka P1, kandidat nije dobio potvrdu o položenom ispitu, obaveštava administrativnog radnika i proces se nastavlja u koraku 7 podtoka P1.
   \end{itemize}

\end{itemize}  

\begin{figure}[H]
  \begin{center}
      \includegraphics[width=140mm, height=70mm]{Diagrams/dijagram_aktivnosti_polaganje_prakticnog_ispita.png}
  \end{center}
  \caption {Dijagram aktivnosti - Polaganje praktičnog ispita}
  \label{activity_polaganje_praktičnog_ispita}

\end{figure}


Na slici \ref{fig:bpmnP_podnosenje_zahteva} je prikazan BPMN dijagram procesa praktične nastave, na slici \ref{fig:bpmnS_podnosenje_zahteva} je prikazan BPMN dijagram saradnje praktične nastave, koji predstavljju sled događaja u sistemu za 4 slučaja upotrebe:
\begin{itemize}
    \item Prijava za praktičnu obuku
    \item Pohađanje praktične obuke
    \item Prijava za praktični ispit
    \item Polaganje praktičnog ispita
\end{itemize}

\begin{figure}[H]
    \begin{center}
        \includegraphics[width=120mm, height=60mm]{Diagrams/bpmnP_prakticna_nastava.png}
    \end{center}
    \caption {BPMN dijagram procesa praktične nastave}
    \label{fig:bpmnP_podnosenje_zahteva}

\end{figure}

\begin{figure}[H]
    \begin{center}
        \includegraphics[width=120mm, height=60mm]{Diagrams/bpmnS_prakticna_nastava.png}
    \end{center}
    \caption {BPMN dijagram saradnje praktične nastave}
    \label{fig:bpmnS_podnosenje_zahteva}

\end{figure}

\subsection {Vođenje evidencije}
Ovde ćemo predstaviti 3 slučaja upotrebe koje obavlja najvećim delom  nadležni za zaposlene i vrši vođenje različitih evidencija.
Pre svega predstavićemo administrativni deo auto škole, vođenje evidencije o invenatru škole i ostalim pokretnostima i nepokretnostima.
Nakon toga vođenje evidencija o zaposlenim kadrovima i njihovom rasporedu rada. Za kraj tu su i evidencije o polaganju ispita, odnosno formiranje 
odgovarajućih zapisnika o tim proverama.

\begin{figure}[H]
    \begin{center}
        \includegraphics[width=120mm, height=60mm]{Diagrams/evidencija.png}
    \end{center}
    \caption {Praktična obuka}
    \label{usecase_praktična obuka}

\end{figure}

\subsubsection{Formiranje evidencija voznog parka}
\label{subsubsec:vozni park}
\begin{itemize}
  \item \textbf{Kratak opis}: Nadležni za zaposlene ima mogućnost da vodi evidenciju svih kola i instruktora, i da vrši promene u inventaru i rasporedu.
  \item \textbf{Učesnici}:
    \begin{itemize}
    \item Nadležni za zaposlene - korisnik sistema koji vrši raspored inventara i evidenciju.
    \end{itemize}
  \item \textbf{Preduslovi}:
    \begin{itemize}
    \item  Nadležni za zaposlene je uspešno ulogovan na sistem auto škole.
    \item  Sistem je u funkciji.
    \item  Nadležni za zaposlene ima pristup internetu.
    \end{itemize}
  \item \textbf{Postuslovi}:
      \begin{itemize}
      \item  Nadležni za zaposlene je ažurirao vozni park i dodelio vozilima instruktore.
      \end{itemize}
  \item \textbf{Osnovni tok}:
      \begin{enumerate}
        \item Nadležni za zaposlene otvara stranicu za uvid  u stanje voznog parka.
        \item Sistem prikazuje trenutni vozni park i informacije koji instruktor upravlja vozilom.
        \item Nadležni pritiska dugme "Izmeni"  na stranici.
        \item Sistem omogućava organizatoru da izmeni podatke u tabeli.
        \item Nadležni za zaposlene unosi izmene.
        \item Nadležni za zaposlene potvrđuje izmene.
        \item Sistem evidentira izmene.
        \item Sistem šalje mejl o izmenama instruktorima ako je došlo do novog rasporeda.
      \end{enumerate}

  \item \textbf{Alternativni tokovi}:
      \begin{itemize}
        \item A1. \textbf{Neuspela validacija.}
        Ukoliko u koraku 5 nadležni u formu o izmenama unese nevalidne podatke polja formulara koja su neispravna će biti crvena.
        Nakon što nadležni ažurira podatke nastavlja se čuvanje izmena. Proces se nastavlja u koraku 6.
      \end{itemize}
      
  \item \textbf{Specijalni zahtevi}:
      \begin{itemize}
        \item Podaci koji se čuvaju za vozila su id auta, marka, reg broj, insturktor, kilometraža. 
      \end{itemize}
\end{itemize}

\begin{figure}[H]
  \begin{center}
      \includegraphics[width=170mm, height=70mm]{Diagrams/evidencija_vozila.png}
  \end{center}
  \caption {Dijagram aktivnosti - Formiranje evidencija voznog parka}
  \label{activity_evidencija_vozila}

\end{figure}

\subsubsection{Formiranje evidencije o zaposlenima}
\label{subsubsec:vozni park}
\begin{itemize}
  \item \textbf{Kratak opis}: Nadležni za zaposlene ima mogućnost da vodi evidenciju kadrova zaposlenih u auto školi 
  (instruktora, predavača, računovođa, administrativnih radnika). Vodi računa o rasporedu rada, radnom vremenu i slobodnim danima zaposlenih.

  \item 
  \item \textbf{Učesnici}:
    \begin{itemize}
    \item Nadležni za zaposlene.
    \end{itemize}
  \item \textbf{Preduslovi}:
    \begin{itemize}
    \item  Nadležni za zaposlene je uspešno ulogovan na sistem auto škole.
    \item  Sistem je dostupan.
    \item  Nadležni za zaposlene ima pristup internetu.
    \end{itemize}
  \item \textbf{Postuslovi}:
      \begin{itemize}
      \item  Nadležni za zaposlene je ažurirao evidenciju o zaposlenim kadrovima u auto školi.
      \item  Nadležni za zaposlene je ažurirao evidenciju o rasporedu rada zaposlenih.
      \end{itemize}
  \item \textbf{Osnovni tok}:
      \begin{enumerate}
        \item Nadležni za zaposlene otvara stranicu za uvid u spisak zaposlenih radnika.
        \item Sistem prikazuje trenutno zaposlene kao i njihov raspored rada za tekuću nedelju ili dan.
        \item Nadležni pritiska dugme "Izmeni" na stranici.
        \item Sistem omogućava organizatoru da izmeni podatke u tabeli ili doda nove.
        \item Nadležni dodaje novog zaposlenog na spisak ili menja radno vreme nekom zaposlenom po potrebi.
        \item Nadležni potvrđuje izmene.
        \item Sistem evidentira izmene.
        \item Sistem šalje mejl o izmenama zaposlenima ako je došlo do novog rasporeda rada za narednu nedelju ili dan.
      \end{enumerate}

  \item \textbf{Alternativni tokovi}:
      \begin{itemize}
        \item A1. \textbf{Neuspela validacija.}
        Ukoliko u koraku 4 nadležni u formu o izmenama unese nevalidne podatke polja formulara koja su neispravna će biti crvena.
        Nakon što nadležni ažurira podatke nastavlja se čuvanje izmena. Proces se nastavlja u koraku 5.
      \end{itemize}

      
  \item \textbf{Dodatne informacije}:
      \begin{itemize}
        \item Podaci o zaposlenom : ime, prezime, jmbg, broj telefona, raspored rada.
      \end{itemize}
\end{itemize}
\subsubsection{Formiranje evidencije o polaganju ispita}
\label{subsubsec:vozni park}
\begin{itemize}
  \item \textbf{Kratak opis}: Nadležni za zaposlene ima mogućnost da vodi evidenciju o polaganju praktičnog ispita. Vodi računa o vremenu kad se održava ispit, instrukoru koji je zadužen za polaganje, kandidatu koji polaže ispit i o anketi koju kandidat popunjava na kraju ispita.

  \item \textbf{Učesnici}:
    \begin{itemize}
    \item Nadležni za zaposlene.
    \end{itemize}
  \item \textbf{Preduslovi}:
    \begin{itemize}
    \item  Nadležni za zaposlene je uspešno ulogovan na sistem auto škole.
    \item  Sistem je dostupan.
    \item  Nadležni za zaposlene ima pristup internetu.
    \end{itemize}
  \item \textbf{Postuslovi}:
      \begin{itemize}
      \item  Nadležni za zaposlene je ažurirao evidenciju o ispitu.
      \item  Nadležni za zaposlene je ažurirao evidenciju o instruktorima, na osnovu anketa.
      \end{itemize}
  \item \textbf{Osnovni tok}:
      \begin{enumerate}
        \item Nadležni za zaposlene otvara stranicu za uvid u spisak prijavljenih kandidata.
        \item Nadležni za zaposlene beleži u sistem prisustvo kandidata.
        \item Nadležni za zaposlene beleži vreme početka ispita i dodeljenog instruktora u sistem.
        \item Sistem pamti podatke.
        \item Nadležni unosi u sistem rezultate ispita.
        \item Nadležni za zaposlene prosleđuje anketu kandidatu.
        \item Nadležni za zaposlene unosi podatke iz ankete i ažurira evidencije o instruktorima.
        \item Sistem potvrđuje izmene.
      \end{enumerate}

  \item \textbf{Alternativni tokovi}:
      \begin{itemize}
        \item A1. \textbf{Kandidat nije položio.}
        Ukoliko je u koraku 5 nadležni uneo u sistem da je kandidat pao vožnju, proces se prekida.
      \end{itemize}
      
  \item \textbf{Dodatne informacije}:
      \begin{itemize}
        \item Kandidat popunjava anketu o svom iskustu u auto školi i svom iskustu sa izabranim instruktoram.
      \end{itemize}
\end{itemize}



\subsection {Vođenje finansija}
Ovde ćemo predstaviti slučajeve upotrebe koje obavlja najvećim delom računovođa u auto školi.
To podrazumeva vođenje evidencije o isplati plata zaposlenima, uplatama kandidata, kao i održavanje tekućeg cenovnika usluga u auto školi.

\subsubsection{Pregled plata zaposlenih}
\label{subsubsec:vozni park}
\begin{itemize}
  \item \textbf{Kratak opis}: Računovođa može da zatraži pregled plata zaposlenih u određenom vremenskom trenutku. 
  Kada dobije željeni izveštaj, može ga sačuvati na sistemu, izmeniti ili odštampati.

  \item \textbf{Učesnici}:
    \begin{itemize}
    \item Računovođa - korisnik sistema koji ima uvid u plate zaposlenih. 
    \end{itemize}
  \item \textbf{Preduslovi}:
    \begin{itemize}
    \item  Računovođa je uspešno ulogovan na sistem auto škole.
    \item  Sistem je u funkciji.
    \item  Računovođa ima pristup internetu.
    \end{itemize}
  \item \textbf{Postuslovi}:
      \begin{itemize}
      \item  Računovođa je dobio izveštaj o zahtevanim platama za zaposlene.
      \item  Računovođa je uspešno izvršio željene izmene plata.
      \end{itemize}
  \item \textbf{Osnovni tok}:
      \begin{enumerate}
        \item Računovođa otvara stranicu za uvid u spisak zaposlenih radnika i njihovih plata u željenom periodu.
        \item Sistem prikazuje trenutno zaposlene kao i njihove plate.
        \item Računovođa bira dugme "Izmeni" da promeni platu ili  "Odštampaj" za dobijanje izveštaja.
        \item Sistem omogućava računovođi da izmeni podatke u tabeli  ili  izvrši štampanje.
        \item Računovođa potvrđuje izmene ili vrši štampanje.
        \item Sistem evidentira izmene.
      \end{enumerate}

  \item \textbf{Alternativni tokovi}:
      \begin{itemize}
        \item A1. \textbf{Ne mogu se dobiti traženi podaci.}
        Ako sistem u koraku 2 ne može iz nekog razloga da prikaže tražene podatke, računovođa se obaveštava da je došlo do problema.
        Proces se nastavlja od koraka 1 osnovnog toka.
        \item A1. \textbf{Neuspela izmena ili štampanje.}
        Ako u koraku 5 štampač nije dostupan ili su uneti podaci nevalidni, računovođa se obaveštava o problemu. Proces se nastavlja unošenjem
        validnih podataka u slučaju izmene i nastavlja se od koraka 4. U slučaju nedostupnosti štampača proces se završava.
      \end{itemize}

      
  \item \textbf{Specijalni zahtevi}:
      \begin{itemize}
        \item Podaci o zaposlenom : ime, prezime, jmbg, broj telefona, plata.
      \end{itemize}
\end{itemize}

\begin{figure}[H]
  \begin{center}
      \includegraphics[width=170mm, height=70mm]{Diagrams/pregled_zaposlenih.png}
  \end{center}
  \caption {Dijagram aktivnosti - Pregled plata zaposlenih}
  \label{activity_pregled_zaposlenih}

\end{figure}

\subsubsection{Pregled cena usluga}
\label{subsubsec:vozni park}
\begin{itemize}
  \item \textbf{Kratak opis}: Računovođa vodi računa o cenana usluga i mogućim popustima

  \item \textbf{Učesnici}:
    \begin{itemize}
    \item Računovođa - korisnik sistema koji ima uvid u cenu i popust za obuku i saopštava ih kandidatu.
    \item Kandidat - korisnik koji donosi podatke računovođi i dobija informaciju o ceni obuke i mogućem popustu.
    \end{itemize}
  \item \textbf{Preduslovi}:
    \begin{itemize}
    \item  Kandidat želi da se upiše u auto školu.
    \end{itemize}
  \item \textbf{Postuslovi}:
      \begin{itemize}
      \item  Kandidatu je određena cena obuke.
      \end{itemize}
  \item \textbf{Osnovni tok}:
      \begin{enumerate}
        \item Kandidat donosi osnovne podatke računovođi.
        \item Računovođa saopštava osnovnu cenu obuke.
        \item Računovođa određuje popust za datog kandidata.
        \item Računovođa saopštava cenu obuke sa popustom.
        \item Kandidat nastavlja sa upisom.
      \end{enumerate}

  \item \textbf{Alternativni tokovi}:
      \begin{itemize}
        \item A1. \textbf{Kandidat ne prihvata cenu obuke:}
        Kandidatu ne odgovara cena obuke. Proces se prekida.
      \end{itemize}

      
  \item \textbf{Specijalni zahtevi}:
      \begin{itemize}
        \item Računovođa određuje popust na osnovu socijalno ekonomskog statusa i na osnovu pripadnosti društveno osetljivim grupama.
      \end{itemize}
\end{itemize}

\begin{figure}[H]
  \begin{center}
      \includegraphics[width=140mm, height=70mm]{Diagrams/dijagram_aktivnosti_cena_obuke.png}
  \end{center}
  \caption {Dijagram aktivnosti - Pregled cena usluga}
  \label{activity_pregled_cena_usluga}

\end{figure}


